% !TEX root = ../project-outline.tex

\chapter{Introduction}
\label{cha:intro}

Most of the text in this example of a master thesis is quote from 'The Extremes of Good and Evil' (Cicero). Besides this text you find some usage examples in the following sections.

\begin{itemize}
	\item A table can be found in Section \ref{sec:results}. This example (Table \ref{tab:confonly}) is only a suggestion. You are allowed to format your tables in your preferred style.
	\item An example of an algorithm is depicted in Section \ref{sec:diag}. Again, you are allowed to use a different style for algorithms, but the style we used to display Algorithm \ref{alg:efficient-lod} looks quite nice.
	\item Chapter \ref{cha:intro} demontrates how to refer to chapters and algorithms and other elements of your thesis.
	\item You should always place definitions, propositions, and whatever might be useful in an appropriate environment.  Examples can be found in section \ref{sec:prelim}.
\end{itemize}

The structure of this draft is \emph{non-binding}. The structure of your thesis - the breakup in chapters, sections, and so on - highly depends on the chosen topic and should be discussed with your adviser. In this draft we made no use of subsections. While subsections might make sense in your own thesis, we believe that subsubsections should be avoided if possible. Notice that we did not break up this template in different parts using the command \verb|\part{}|. It depends on your own style and your work wether to use this option.

If you cite something, do it in the following way. 
\begin{itemize}
	\item Conference Proceedings: This problem is typically addressed by approaches for selecting the optimal matcher based on the nature of the matching task and the known characteristics of the different matching systems. Such an approach is described in \cite{mochol08matcher}.
	\item Journal Article: S-Match, described in \cite{giunchiglia2008semanticmatching}, employs sound and complete reasoning procedures. Nevertheless, the underlying semantic is restricted to propositional logic due to the fact that ontologies are interpreted as tree-like structures.
	\item Book: According to Euzenat and Shvaiko \cite{euzenat07matcherbook}, we define a correspondence as follows.
\end{itemize}
These are some randomly chosen examples from other works. Take a look at the end of this thesis so see how the bibliography is included.

In this examples thesis you will find two chapters in the appendix. Appendix \ref{cha:appendix-a} describes the program code that might have been part of your work. It depends of the type of work wether such an appendix makes sense. Appendix \ref{cha:appendix-b} contains some additional experimental results. It might happen that most of your experimental results are presented in aggregated form; a complete listing of detailed results in the appendix might make sense. Nevertheless, there are no hard requirements with respect to the use of an appendix. It is up to you wether or not you will use an appendix (well .. as long as your adviser does not tell you something else).

Some words about the list of figures, list of algorithm, and so on. Listing your figures is obligatory. It depends on your own choice, wether to include a list of other \textit{things}. Relevant aspects are the subject of your thesis and the way you develop your ideas. For example: If your work contains lots of tables with different experimental results, add a list of tables. If you develop and explicitly state algorithms, add a list of algorithms. 

\emph{Very Important:} Do not forget to sign (manually) the last page, before you submit/deliver the final version of your thesis. Otherwise your work cannot be accepted for legal reasons.
 
\section{Problem Statement}
 
But I must explain to you how all this mistaken idea of denouncing pleasure and praising pain was born and I will give you a complete account of the system, and expound the actual teachings of the great explorer of the truth, the master-builder of human happiness. No one rejects, dislikes, or avoids pleasure itself, because it is pleasure, but because those who do not know how to pursue pleasure rationally encounter consequences that are extremely painful. Nor again is there anyone who loves or pursues or desires to obtain pain of itself, because it is pain, but because occasionally circumstances occur in which toil and pain can procure him some great pleasure. To take a trivial example, which of us ever undertakes laborious physical exercise, except to obtain some advantage from it? But who has any right to find fault with a man who chooses to enjoy a pleasure that has no annoying consequences, or one who avoids a pain that produces no resultant pleasure?

On the other hand, we denounce with righteous indignation and dislike men who are so beguiled and demoralized by the charms of pleasure of the moment, so blinded by desire, that they cannot foresee the pain and trouble that are bound to ensue; and equal blame belongs to those who fail in their duty through weakness of will, which is the same as saying through shrinking from toil and pain. These cases are perfectly simple and easy to distinguish. In a free hour, when our power of choice is untrammelled and when nothing prevents our being able to do what we like best, every pleasure is to be welcomed and every pain avoided. But in certain circumstances and owing to the claims of duty or the obligations of business is will frequently occur that pleasures have to be repudiated and annoyances accepted. The wise man therefore always holds in these matters to this principle of selection: he rejects pleasures to secure other greater pleasures, or else he endures pains to avoid worse pains. 

\section{Contribution}

But I must explain to you how all this mistaken idea of denouncing pleasure and praising pain was born and I will give you a complete account of the system, and expound the actual teachings of the great explorer of the truth, the master-builder of human happiness. No one rejects, dislikes, or avoids pleasure itself, because it is pleasure, but because those who do not know how to pursue pleasure rationally encounter consequences that are extremely painful. Nor again is there anyone who loves or pursues or desires to obtain pain of itself, because it is pain, but because occasionally circumstances occur in which toil and pain can procure him some great pleasure. To take a trivial example, which of us ever undertakes laborious physical exercise, except to obtain some advantage from it? But who has any right to find fault with a man who chooses to enjoy a pleasure that has no annoying consequences, or one who avoids a pain that produces no resultant pleasure?

On the other hand, we denounce with righteous indignation and dislike men who are so beguiled and demoralized by the charms of pleasure of the moment, so blinded by desire, that they cannot foresee the pain and trouble that are bound to ensue; and equal blame belongs to those who fail in their duty through weakness of will, which is the same as saying through shrinking from toil and pain. These cases are perfectly simple and easy to distinguish. In a free hour, when our power of choice is untrammelled and when nothing prevents our being able to do what we like best, every pleasure is to be welcomed and every pain avoided. But in certain circumstances and owing to the claims of duty or the obligations of business is will frequently occur that pleasures have to be repudiated and annoyances accepted. The wise man therefore always holds in these matters to this principle of selection: he rejects pleasures to secure other greater pleasures, or else he endures pains to avoid worse pains. 

\section{Related Work}

But I must explain to you how all this mistaken idea of denouncing pleasure and praising pain was born and I will give you a complete account of the system, and expound the actual teachings of the great explorer of the truth, the master-builder of human happiness. No one rejects, dislikes, or avoids pleasure itself, because it is pleasure, but because those who do not know how to pursue pleasure rationally encounter consequences that are extremely painful. Nor again is there anyone who loves or pursues or desires to obtain pain of itself, because it is pain, but because occasionally circumstances occur in which toil and pain can procure him some great pleasure. To take a trivial example, which of us ever undertakes laborious physical exercise, except to obtain some advantage from it? But who has any right to find fault with a man who chooses to enjoy a pleasure that has no annoying consequences, or one who avoids a pain that produces no resultant pleasure?

On the other hand, we denounce with righteous indignation and dislike men who are so beguiled and demoralized by the charms of pleasure of the moment, so blinded by desire, that they cannot foresee the pain and trouble that are bound to ensue; and equal blame belongs to those who fail in their duty through weakness of will, which is the same as saying through shrinking from toil and pain. These cases are perfectly simple and easy to distinguish. In a free hour, when our power of choice is untrammelled and when nothing prevents our being able to do what we like best, every pleasure is to be welcomed and every pain avoided. But in certain circumstances and owing to the claims of duty or the obligations of business is will frequently occur that pleasures have to be repudiated and annoyances accepted. The wise man therefore always holds in these matters to this principle of selection: he rejects pleasures to secure other greater pleasures, or else he endures pains to avoid worse pains. 