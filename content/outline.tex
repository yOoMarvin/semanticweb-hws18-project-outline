\section{Problem Statement}
Everyone who does regular city trips knows the problem. You who visit a foreign city for and just don’t have the same experience as people who live there.  A lot of information sources, like tourist centers, give you a very biased opinion on what you should see and what not. But what if somebody wants to look for something special or has special interests? If somebody really wants to get to know a city and the different districts,  this takes a lot of time and preparation.
\\ \\
This problem should be solved within this project through an application that uses semantic web technologies.

\section{Application Goal}
The overall goal of the application is to give the user a visual help to easily identify the particularities of the different city districts. This should be done with the help of a map as the main user interface where the different districts are marked in different colors in the form of rectangles.
\\
Examples of this would be categories like \textit{museums, science, students} and a lot more. The detailed categories depend on the districts and offer of the respective city. 
\\ \\
In the end, the user can visit the web application where he/she sees a map of a city where different special districts are marked. Therefore trips can be planned more individual and efficient.

\section{Datasets}

\section{Techniques}

\section{Evaluation}


