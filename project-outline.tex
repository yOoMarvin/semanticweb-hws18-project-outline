% do not change these two lines (this is a hard requirement
% there is one exception: you might replace oneside by twoside in case you deliver 
% the printed version in the accordant format
\documentclass[11pt,titlepage,oneside,openany]{article}
\usepackage{times}


\usepackage{graphicx}
\usepackage{latexsym}
\usepackage{amsmath}
\usepackage{amssymb}

\usepackage{ntheorem}

% \usepackage{paralist}
\usepackage{tabularx}

% this packaes are useful for nice algorithms
\usepackage{algorithm}
\usepackage{algorithmic}

% well, when your work is concerned with definitions, proposition and so on, we suggest this
% feel free to add Corrolary, Theorem or whatever you need
\newtheorem{definition}{Definition}
\newtheorem{proposition}{Proposition}


% its always useful to have some shortcuts (some are specific for algorithms
% if you do not like your formating you can change it here (instead of scanning through the whole text)
\renewcommand{\algorithmiccomment}[1]{\ensuremath{\rhd} \textit{#1}}
\def\MYCALL#1#2{{\small\textsc{#1}}(\textup{#2})}
\def\MYSET#1{\scshape{#1}}
\def\MYAND{\textbf{ and }}
\def\MYOR{\textbf{ or }}
\def\MYNOT{\textbf{ not }}
\def\MYTHROW{\textbf{ throw }}
\def\MYBREAK{\textbf{break }}
\def\MYEXCEPT#1{\scshape{#1}}
\def\MYTO{\textbf{ to }}
\def\MYNIL{\textsc{Nil}}
\def\MYUNKNOWN{ unknown }
% simple stuff (not all of this is used in this examples thesis
\def\INT{{\mathcal I}} % interpretation
\def\ONT{{\mathcal O}} % ontology
\def\SEM{{\mathcal S}} % alignment semantic
\def\ALI{{\mathcal A}} % alignment
\def\USE{{\mathcal U}} % set of unsatisfiable entities
\def\CON{{\mathcal C}} % conflict set
\def\DIA{\Delta} % diagnosis
% mups and mips
\def\MUP{{\mathcal M}} % ontology
\def\MIP{{\mathcal M}} % ontology
% distributed and local entities
\newcommand{\cc}[2]{\mathit{#1}\hspace{-1pt} \# \hspace{-1pt} \mathit{#2}}
\newcommand{\cx}[1]{\mathit{#1}}
% complex stuff
\def\MER#1#2#3#4{#1 \cup_{#3}^{#2} #4} % merged ontology
\def\MUPALL#1#2#3#4#5{\textit{MUPS}_{#1}\left(#2, #3, #4, #5\right)} % the set of all mups for some concept
\def\MIPALL#1#2{\textit{MIPS}_{#1}\left(#2\right)} % the set of all mips


% custom stuff
\usepackage{hyperref}


\begin{document}

\pagenumbering{roman}
% lets go for the title page, something like this should be okay
\begin{titlepage}
	\vspace*{2cm}
  \begin{center}
   {\huge Project Outline: ... \\}
   \vspace{2cm} 
   {\Large Student Project Semantic Web Technologies HWS18\\}
   \vspace{2cm}
   {\Large Presented by \\}
   \vspace{0.5cm}
    {
    Nele Ecker\\
    Alex L\"utke\\
    Marvin Messenzehl \\
   }
   \vspace{1cm} 
   { Submitted to the\\
    Data and Web Science Group\\
    Prof.\ Dr.\ Heiko Paulheim \& Sven Hertling \\
    University of Mannheim\\} \vspace{2cm}
   {October - December 2018}
  \end{center}
\end{titlepage} 

% no lets make some add some table of contents
\tableofcontents
\newpage

%\listofalgorithms

%\listoffigures

%\listoftables

% evntuelly you might add something like this
% \listtheorems{definition}
% \listtheorems{proposition}

%\newpage


% okay, start new numbering ... here is where it really starts
\pagenumbering{arabic}

%%%%%%%%%%%%%%%%%%%%%%%%%%%%%%%%%%%%%%%%

% INPUTS
\section{Problem Statement}
Everyone who does regular city trips knows the problem. You who visit a foreign city for and just don’t have the same experience as people who live there.  A lot of information sources, like tourist centers, give you a very biased opinion on what you should see and what not. But what if somebody wants to look for something special or has special interests? If somebody really wants to get to know a city and the different districts,  this takes a lot of time and preparation.
\\ \\
This problem should be solved within this project through an application that uses semantic web technologies.

\section{Application Goal}
The overall goal of the application is to give the user a visual help to easily identify the particularities of the different city districts. This should be done with the help of a map as the main user interface where the different districts are marked in different colors in the form of rectangles.
\\
Examples of this would be categories like \textit{museums, science, students} and a lot more. The detailed categories depend on the districts and offer of the respective city. 
\\ \\
In the end, the user can visit the web application where he/she sees a map of a city where different special districts are marked. Therefore trips can be planned more individual and efficient.

\section{Datasets}

\section{Techniques}

\section{Evaluation}






\newpage


\pagestyle{empty}


%\section*{Ehrenw\"ortliche Erkl\"arung}
%Ich versichere, dass ich die beiliegende Master-/Bachelorarbeit ohne Hilfe Dritter
%und ohne Benutzung anderer als der angegebenen Quellen und Hilfsmittel
%angefertigt und die den benutzten Quellen w\"ortlich oder inhaltlich
%entnommenen Stellen als solche kenntlich gemacht habe. Diese Arbeit
%hat in gleicher oder \"ahnlicher Form noch keiner Pr\"ufungsbeh\"orde
%vorgelegen. Ich bin mir bewusst, dass eine falsche Er- kl\"arung rechtliche Folgen haben
%wird.
%\\
%\\

%\noindent
%Mannheim, den 31.08.2014 \hspace{4cm} Unterschrift

\end{document}
